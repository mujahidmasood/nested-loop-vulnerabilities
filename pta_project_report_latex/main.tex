\documentclass[authoryear,preprint]{sigplanconf}

\usepackage{amsmath}
\usepackage{amsfonts}
\usepackage{amssymb}
\usepackage{amsthm}
\usepackage{algorithm2e}
\usepackage{listings}
\usepackage{xcolor}
\usepackage{tikz}
\usepackage{booktabs}
\usepackage{subfigure}
\usepackage[english]{babel}
\usepackage{blindtext}
\usetikzlibrary{shapes.geometric, arrows,chains}
\bibliographystyle{abbrvnat}
\usepackage{url}
\usepackage{hyperref}

\usepackage{color}

\definecolor{dkgreen}{rgb}{0,0.6,0}
\definecolor{gray}{rgb}{0.5,0.5,0.5}
\definecolor{mauve}{rgb}{0.58,0,0.82}
\usepackage{enumitem}
\newlist{steps}{enumerate}{1}
\setlist[steps, 1]{label = Step \arabic*:}

\lstset{frame=tb,
  language=Java,
  aboveskip=3mm,
  belowskip=3mm,
  showstringspaces=false,
  columns=flexible,
  basicstyle={\small\ttfamily},
  numbers=left,
  numberstyle=\tiny\color{gray},
  keywordstyle=\color{blue},
  commentstyle=\color{dkgreen},
  stringstyle=\color{mauve},
  breaklines=true,
  breakatwhitespace=true,
  tabsize=3
}

\begin{document}

\special{papersize=a4}
\setlength{\pdfpageheight}{\paperheight}
\setlength{\pdfpagewidth}{\paperwidth}


\title{Identifying Algorithmic Complexity Vulnerabilities Caused by
Input-Dependent Nested Loops}

\authorinfo{Mujahid Masood,Rafique Nazir}{mujahid.masood@stud.tu-darmstadt.de,rafique.nazir@stud.tu-darmstadt.de}{}
\maketitle

\section{Problem Statement}
\label{sec:problemstatement}

The single-threaded event model of JavaScript makes it vulnerable
to a specific class of denial of service attack called algorithmic
complexity attacks. These attacks consist of exploiting the worst
case performance of algorithms to trigger slow computations that
block the event loop for a large period of time. The focus of this
project is to
\begin{itemize}
\item Identify the functions in code.
\item Identify the input parameters to the functions.
\item Identify the loops using input parameters of functions.
\item Impact of using function input parameters with loops in terms
of execution time as well as denial of service attack.
\end{itemize}


\section{Input dependent loops and execution time}
\label{sec:introduction}
Execution time of the functions is really important to write scalable applications. In normal application overall execution time of function depend on the slowest function.

Consider the code in Listing \ref{l:iterate} we have iterate function which has max as input parameter, it has one loop which is iterating up till max.

So caller of function can pass input of \begin{math} 10_{10} \end{math} and execution time of simple function with 1 loop will be around \textit{11s}.

\begin{lstlisting}[caption=iterate function with 1 loop,label=l:iterate,language=Java]

function iterate(max){

    var start = process.hrtime();
    var precision = 3;
    for(var i = 0; i< max; i++){
        var c = 10 + 5;
    }

    var elapsed = process.hrtime(start)[1] / 1000000;
    
	console.log(
		process.hrtime(start)[0] + " s, "+
		elapsed.toFixed(precision)+" ms - " 
	);
}

\end{lstlisting}

If other functions in application are taking less time, iterate
function is subject to performance bug and also security bug specially
in node modules.

This problems gets interesting if we introduce nested loops
i.e, 2 or 3 nested loops. Consider the code in Listing 2. Function
double loop has input parameters max and 2 nested loops. Client
of doubleLoop can pass input max 105
and can introduce as delay
of around 6 s, 491.041 ms . Important thing to note is function is
only doing simple sum of 10 and 5 but due to input dependent loop
execution time of function takes much time.

\begin{lstlisting}[caption=doubleLoop function with 2 nested loop,label=l:doubleLoop,language=Java]

function doubleLoop(max){

    var start = process.hrtime();
    var precision = 3;
    for(var i = 0; i< max; i++){
    	for(var j=0; j< i; j++){
			 var c = 10 + 5;
 		}
 	}

    var elapsed = process.hrtime(start)[1] / 1000000;
	console.log(
		process.hrtime(start)[0] + " s, "+
		elapsed.toFixed(precision)+" ms - " 
	);
}

doubleLoop(Math.pow(10,5));

\end{lstlisting}

As we increase the nested loops input to the parameter gets
smaller. A function with 3 nested loops and input parameter max
can have execution time of around 6 s with max = 103.2
Consider the code in Listing 3.

\begin{lstlisting}[caption=tripleLoop function with 2 nested loop,label=l:tripleLoop,language=Java]

function tripleLoop(max){

    var start = process.hrtime();
    var precision = 3;
    for(var i = 0; i< max; i++){
    	for(var j=0; j< i; j++){
    		for(var k=0; k< max; k++){
				 var c = 10 + 5;
 			}
 		}
 	}

    var elapsed = process.hrtime(start)[1] / 1000000;
	console.log(
		process.hrtime(start)[0] + " s, "+
		elapsed.toFixed(precision)+" ms - " 
	);
}

tripleLoop(Math.pow(10,3.2));

\end{lstlisting}


\section{Problems with using loops depending on function input parameters}
\label{sec:problems}
Code in Listing \ref{l:doubleLoop} and Listing \ref{l:tripleLoop} are not only subject to performance issues but also to denial of service (DoS) attacks. Attacker can control the input parameter and can introduce delay of 10-15 seconds in the execution time of function. Node.js security experts consider any slowdown larger than one second as security relevant.

\section{Avoiding the problem}
\label{sec:avoiding}

Following are some approaches which can be used to avoid the problem with input dependent nested loops.

\subsection{Approach 1 : Introduce Upper bound on input parameters}
One simple way to avoid such problem is introducing upper bound on input parameter.
Consider the code in Listing \ref{l:avoidProblem} which exits from the function if input exceeds the given bound which is \begin{math} 10^{2} \end{math}.

\begin{lstlisting}[caption=avoidProblem function with 3 nested loop,label=l:avoidProblem,language=Java]

function avoidProblem(max){
	if(max > Math.pow(10,2)) {
		return;	
	}
    for(var i = 0; i< max; i++){
    	for(var j=0; j< i; j++){
    		for(var k=0; k< max; k++){
				 var c = 10 + 5;
 			}
 		}
 	}
}

avoidProblem(Math.pow(10,3.2));

\end{lstlisting}

\subsubsection{Problems with Approach 1}

Introducing such input bound checks as in Listing \ref{l:avoidProblem} requires full understanding of the code and also the usage of function.

\subsection{Approach 2 : Changing the logic} 
Other way can be changing the logic from nested loops to maybe introducing new functions which means one needs to find places where input dependent functions with loops are used in the code.
\subsubsection{Problems with Approach 2}
\begin{itemize}
\item Code is already deployed in production environment.
\item JavaScript uses minified versions of files.
\item JavaScript file can have 10000s lines of code.
\item Different variations of functions.
		\begin{itemize}
			\item JavaScript can use assignment of function to other variable 
				e.g, var a = function() {}
			\item Using anonymous functions
				e.g, (function())
			\item JavaScript also uses nested functions
				e.g, (function(a, function(){}))
		\end{itemize}
\item Different variations of Loops
	 e.g  For, While, For In, forEach etc.
\item Assignment of function input parameter to other variables and using that in loop.
\item Using input parameter in function call which in turn uses for loop.
	 
			
\end{itemize}




\citep{Aho86-Compilers}
\bibliography{references}

\end{document}